\subsection{Cryo-recovery protocol}

When the gas system needs to be stopped for an intervention in the detector the gas Xenon needs to be recovered and saved. The way of recovering the Xenon from the vessel and the gas system is with cryogenics.

NEXT-DEMO gas system has two spherical bottles specially designed for this use (see attachements) that are connected to the vessel and to the gas system. The bottles are placed in a thermally insolated container that is filled with liquid nitrogen until it reaches (approx.) half of the height of the bottles. Once the nitrogen stops boiling we open the corresponding valve to the part of the system that we want to recover, vessel or gas system. It is also possible to have both open at the same time if needed. Then the Xenon flows into the recovery bottles where it frezzes.

Once the pressure in the vessel/gas system is stable at ~0.03 absolute bar the bottles are full and we close them. Then we need to wait for the liquid nitrogen to evaporate.